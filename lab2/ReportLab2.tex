\documentclass[12pt,a4paper]{article}
\usepackage[utf8]{inputenc}
\usepackage[ukrainian]{babel}
\usepackage{amsmath}
\usepackage{amsfonts}
\usepackage{amssymb}
\usepackage{graphicx}
\usepackage{hyperref}
\usepackage[left=2cm,right=2cm,top=2cm,bottom=2cm]{geometry}

\begin{document}

\begin{titlepage}
    \centering
    \vspace*{1cm}

    \Large
    Київський національний університет імені Тараса Шевченка \\

    \vspace{0.5cm}

    \large
    Факультет комп'ютерних наук та кібернетики \\

    \vspace{0.5cm}

    Кафедра інтелектуальних інформаційних систем \\

    \vspace{0.5cm}

    Алгебро-автоматичні методи проектування програмного забезпечення \\

    \vspace{3cm}

    \textbf{Лабораторна робота 2} \\

    \vspace{0.5cm}

    Скінченні автомати \\
    Аналіз автоматів Мілі: побудова регулярних виразів мови, які акцептуються автоматом Мілі. \\

    \vspace{2cm}

    Виконали студенти 1-го курсу \\

    \vspace{0.2cm}

    Групи ПЗС-1 \\

    \vspace{0.1cm}

    Рябов Кирило \\

    \vspace{0.1cm}

    Соколов Михайло \\

    \vspace{0.1cm}

    Рибачок Руслан \\

    \vfill

    2023

\end{titlepage}

\section*{Лабораторна 2: Аналіз автоматів Мілі: побудова регулярних виразів мови, які акцептуються автоматом Мілі.}

\textbf{Мета:} Побудувати регулярні вирази мови, що акцептуються автоматом Мілі.

\section*{Хід роботи.}

Було реалізовано два алгоритми для регулярних виразів мови, які акцептуються автоматом Мілі:
\begin{enumerate}
  \item Конвертація в автомат Мура, та аналіз СА шляхом елімінації станів.
  \item Аналіз через систему рівнянь, що складається за таблицями переходів та виходів.
\end{enumerate}

\subsection*{Посилання на репозиторій з кодом:}

\href{https://github.com/KyrylR/Automata-1-2023-labs}{https://github.com/KyrylR/Automata-1-2023-labs}

\subsection*{Псевдокод для першого підходу:}

\begin{flalign*}
& \textbf{Вхід:} \quad \text{Автомат Мілі } A = (A, X, Y, f, a_0, F) & \\
& \textbf{Вихід:} \quad \text{ Регулярні вирази мови, що акцептуються автоматом Мілі } & \\
& \textbf{Кроки:} &
\end{flalign*}

\begin{enumerate}
    \item \textbf{Конвертуємо в автомат Мура за алгоритмом:}

    \includegraphics[width=1\textwidth]{mealy_moor.png}

    \item Вилучаємо всі позначки станів автомата В і оголошуємо заключними всі ті стани, які були позначені символами із множини М. Після цього отримуємо скінченний ініціальний Х-автомат, до якого можна застосовувати алгоритм згаданий нижче.

    \item Зауважимо, що після того як задача аналізу стане розв’язаною для автомата Мура, то вилучаємо із отриманої мови пусте слово, якщо воно їй належить

    \item \textbf{Аналізуємо СА шляхом елімінації станів:}
    \begin{enumerate}
        \item Для кожного проміжного стану, відмінного від початкового і заключних станів, застосовуємо редукції, елімінуючи ці стани
        \item Якщо \( z != 0 \), то отримуємо в кінці автомат з двома станами, що відповідає регулярному виразу: \( (R \lor SU^*T)^*SU^* \)
        \item Якщо \( a0 \) є заключним станом, то в результаті редукції отримуємо такий автомат якому відповідає регулярний вираз \( R^* \)
        \item Остаточний регулярний вираз знаходимо як об’єднання всіх регулярних виразів, отриманих за допомогою редукцій станів автомата, за всіма заключними станами автомата А.

    \end{enumerate}

    \item \textbf{Повернутаємо результуючий регулярний вираз \( B \).}
\end{enumerate}

\subsection*{Псевдокод для другого підходу:}

\begin{flalign*}
& \textbf{Вхід:} \quad \text{Автомат Мілі } A = (A, X, Y, f, a_0, F) & \\
& \textbf{Вихід:} \quad \text{ Регулярні вирази мови, що акцептуються автоматом Мілі } & \\
& \textbf{Кроки:} &
\end{flalign*}

\begin{enumerate}
        \item Маємо автомат Мілі, визначений множиною станів, вхідним алфавітом, набором переходів і функцією виводу, а також множиною фінальних символів.
        \item Виводимо деталі автомата Мілі.
        \item Будуємо набір лівих лінійних рівнянь на основі переходів автомата:
        \begin{itemize}
            \item Для кожного переходу в автоматі:
            \begin{itemize}
                \item Для кожного символу в значенні переходу:
                \begin{itemize}
                    \item Якщо у стану-ключа для символу немає рівняння, створюємо нове рівняння для цього стану.
                    \item Додаємо символ до рівняння стану-ключа.
                \end{itemize}
            \end{itemize}
        \end{itemize}
        \item Розв'язуємо ліволінійні рівняння.
        \item Будуємо набір рівнянь SM на основі переходів автомата, які ведуть до станів, асоційованих з фінальними символами:
        \begin{itemize}
            \item Для кожного переходу в автоматі:
            \begin{itemize}
                \item Для кожного символу в значенні переходу:
                \begin{itemize}
                    \item Якщо символ виводу знаходиться в множині фінальних символів:
                    \begin{itemize}
                        \item Якщо у стану-ключа для символу немає рівняння, створюємо нове рівняння для цього стану.
                        \item Додаємо символ до рівняння стану-ключа.
                    \end{itemize}
                \end{itemize}
            \end{itemize}
        \end{itemize}
        \item Розв'язуємо рівняння SM
\end{enumerate}

\newpage

\subsection*{Складність алгоритмів:}
Як випливає з наведених вище кроків, перший підхід має таку оцінку  \( O(|A|^3|X| + |A||Y||X|) \), де \( |A||Y||X|) \) оцінка конвертації Мілі-Мура.
\\
Другий підхід має таку оцінку \( O(2|A||X||f| + n^3) \), де \( O(2|A||X||f|) \) потрібно для побудови рівнянь, а \( O(n^3) \) для вирішення системи лінійних рівннянь.

\subsection*{Результат роботи алгоритму:}

Алгоритм для перетворення е-НДСА на НДСА дозволяє нам ефективно обробляти epsilon-переходи та враховувати їх в результуючому НДСА. Це допомагає уникнути неоднозначності, яка може виникнути через присутність epsilon-переходів.

\vspace{1em}
\textbf{Приклад 1:}
\vspace{0.5em}

Дано наступний автомат Міллі:
\\
\includegraphics[width=0.7\textwidth]{Mealy2.png}

\textbf{Результат роботи 1 підходу:}

\begin{enumerate}
    \item Отриманий автомат Мура: \\
    \includegraphics[width=0.7\textwidth]{MoorFromMealy2.png}
    \item Результат виконання програми \\
    \includegraphics[width=0.7\textwidth]{res1.2.png}
\end{enumerate}

\textbf{Результат роботи 2 підходу:}

\includegraphics[width=0.9\textwidth]{2_system.png}

\\

\vspace{1em}
\textbf{Приклад 2:}
\vspace{0.5em}

Дано наступний автомат Міллі:
\\
\includegraphics[width=0.7\textwidth]{Mealy1.png}

\textbf{Результат роботи 1 підходу:}

\begin{enumerate}
    \item Отриманий автомат Мура: \\
    \includegraphics[width=0.7\textwidth]{MoorFromMealy.png}
    \item Результат виконання програми \\
    \includegraphics[width=0.7\textwidth]{res2.2.png}
\end{enumerate}

\textbf{Результат роботи 2 підходу:}

\includegraphics[width=0.9\textwidth]{1_system.png}

\\

\vspace{1em}
\textbf{Висновок:}
\vspace{0.5em}

У даній лабораторній роботі були реалізовані два алгоритми для побудови регулярних виразів мови, які акцептуються автоматом Мілі.

Перший підхід полягає у конвертації автомата Мілі в автомат Мура та аналізі СА шляхом елімінації станів. Другий підхід використовує систему рівнянь, яка складається на основі таблиць переходів та виходів.

Складність першого підходу становить \(O(|A|^3|X| + |A||Y||X|)\), де \(|A|\) - кількість станів, \(|X|\) - кількість символів вхідного алфавіту, \(|Y|\) - кількість символів вихідного алфавіту. Складність другого підходу становить \(O(2|A||X||f| + n^3)\), де \(|f|\) - кількість функцій виводу, а \(n\) - загальна кількість елементів у системі рівнянь.

Кожен підхід має свої переваги і недоліки, і вибір між ними залежить від конкретних умов задачі. Незважаючи на те, що обидва підходи ведуть до однакового результату, другий підхід може бути більш ефективним при великій кількості станів і символів в алфавітах, оскільки він не вимагає конвертації автомата Мілі в автомат Мура.


\end{document}
