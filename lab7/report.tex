\documentclass[12pt,a4paper]{article}
\usepackage[utf8]{inputenc}
\usepackage[ukrainian]{babel}
\usepackage{amsmath}
\usepackage{amsfonts}
\usepackage{amssymb}
\usepackage{graphicx}
\usepackage{hyperref}
\usepackage[left=2cm,right=2cm,top=2cm,bottom=2cm]{geometry}

\usepackage{tikz}
\usetikzlibrary{automata, positioning, arrows.meta}

% Define custom command for text reference
\newcommand{\textref}[2]{\hyperref[#1]{#2}}
\newcommand{\s}[1]{\texttt{#1}}

\begin{document}

\begin{titlepage}
    \centering
    \vspace*{1cm}
    
    \Large
    Київський національний університет імені Тараса Шевченка \\
    
    \vspace{0.5cm}
    
    \large
    Факультет комп'ютерних наук та кібернетики \\
    
    \vspace{0.5cm}
    
    Кафедра інтелектуальних інформаційних систем \\
    
    \vspace{0.5cm}
    
    Алгебро-автоматичні методи проектування програмного забезпечення \\
    
    \vspace{3cm}
    
    \textbf{Лабораторна робота 7} \\
    
    \vspace{0.5cm}
    
    Скінченні автомати \\
    За граматикою побудувати магазинний автомат, який розпізнає дану граматику.
    
    \vspace{2cm}
    
    Виконали студенти 1-го курсу \\

    \vspace{0.2cm}
    
    Групи ПЗС-1 \\

    \vspace{0.1cm}
    
    Рябов Кирило \\

    \vspace{0.1cm}
    
    Соколов Михайло \\

    \vspace{0.1cm}
    
    Рибачок Руслан \\
    
    \vfill
    
    2023
    
\end{titlepage}

\section*{Лабораторна 7: За граматикою побудувати магазинний автомат (МА), який розпізнає дану граматику }

\section*{Мета}
Основною метою цієї роботи є побудова та опис автомата, який розпізнає мову, породжену заздалегідь визначеною контекстно-вільною граматикою. Це передбачає формальне визначення МА, яке відповідає правилам породження граматики, гарантуючи, що автомат точно приймає всі і тільки рядки в мові.

\section*{Хід роботи}

\subsection*{Постановка задачі}
У цій роботі розглянуто задачу побудови магазинного автомату на основі наступної контекстно-вільної граматики \( G \):

\[ G = (\{a, b\}, \{\sigma, \beta\}, \sigma, P) \]

з відповідними правилами \( P \), визначеними наступним чином:

\[ P = \left\{ \begin{array}{cl}
\sigma & \rightarrow \sigma a \;|\; a \\
\beta & \rightarrow \beta \beta \;|\; b \\
\end{array} \right. \]

Мета полягає у синтезі МА, який приймає мову \( L(G) \), яка складається з усіх рядків, що є похідними від \( G \).

\subsection*{Побудова МА}
МА визначено як сімка \( (Q, \Sigma, \Gamma, \delta, q_0, Z, F) \) де:

\begin{itemize}
    \item \( Q \) - скінченна множина станів \(\{q_{start}, q_{read}, q_{build}, q_{accept}, q_{reject}\}\).
    \item \( \Sigma \) - вхідний алфавіт граматики \(\{a, b\}\).
    \item \( \Gamma \) - алфавіт стеку \(\{\$, A, B\}\).
    \item \( \delta \) - функція переходу, описана нижче.
    \item \( q_0 = q_{start} \) - початковий стан.
    \item \( Z = \$ \) - початковий символ стеку.
\item \( F = \{q_{accept}\} \) - множина станів прийняття.
\end{itemize}

Стани та переходи МА визначаються наступним чином:

\begin{enumerate}
    \item Автомат починає роботу у стані \( q_{start} \), переходить у стан \( q_{read} \) і виштовхує у стек початковий символ стеку \( \$ \).
    \item У стані \( q_{read} \), для кожного 'a', прочитаного з порожнім стеком або з 'A' нагорі, автомат виштовхує 'A' до стеку, відзначаючи прочитання символу 'a'.
    \item У стані \( q_{read} \), для кожного 'b', прочитаного з порожнім стеком, автомат виштовхує 'B' на стек. Якщо 'b' зчитується з 'B' на стеку, він виштовхує (pop) 'B', гарантуючи парну кількість 'b'.
    \item Якщо 'b' зчитується з 'A' на стеку, або 'a' чи '\( \epsilon \)' з 'B' нагорі стеку, автомат переходить до \( q_{reject} \), оскільки рядок не відповідає граматиці.
    \item Автомат переходить до \( q_{build} \), як тільки зустрічає кінець вхідного рядка. У \( q_{build} \) для кожного прочитаного символу він витягує зі стеку 'A', що означає завершення розбору рядка.
    \item Автомат переходить до \( q_{accept} \) з \( q_{build} \), коли стек очищується до початкового символу стеку \( \$ \), що означає, що рядок відповідає правилам граматики.
    \item Стан \( q_{reject} \) є неприйнятним станом, що означає, що вхідний рядок не приймається автоматом.
\end{enumerate}

Ця конструкція гарантує, що МА точно розпізнає рядки з 'a', за якими слідує парна кількість 'b', як це визначено граматикою.

\subsection*{Табличне представлення переходів}
Для наочності функцію переходу \( \delta \) можна подати у табличному форматі:

\begin{table}[h]
\centering
\begin{tabular}{c c c c c c}
\hline
Current State & Input Symbol & Stack Top & Next State & Stack Operation \\
\hline
\( q_{start} \) & \( \epsilon \) & \( \epsilon \) & \( q_{read} \) & Push \( \$ \) \\
\( q_{read} \) & \( a \) & \( \epsilon \) & \( q_{read} \) & Push \( A \) \\
\( q_{read} \) & \( a \) & \( A \) & \( q_{read} \) & Push \( A \) \\
\( q_{read} \) & \( b \) & \( \epsilon \) & \( q_{read} \) & Push \( B \) \\
\( q_{read} \) & \( b \) & \( B \) & \( q_{read} \) & Pop \\
\( q_{read} \) & \( a \) & \( B \) & \( q_{reject} \) & No operation \\
\( q_{read} \) & \( b \) & \( A \) & \( q_{reject} \) & No operation \\
\( q_{read} \) & \( \epsilon \) & \( B \) & \( q_{reject} \) & No operation \\
\( q_{read} \) & \( \epsilon \) & \( A \) & \( q_{read} \) & Pop \\
\( q_{read} \) & \( \epsilon \) & \( \epsilon \) & \( q_{build} \) & No operation \\
\( q_{build} \) & \( \epsilon \) & \( A \) & \( q_{build} \) & Pop \\
\( q_{build} \) & \( \epsilon \) & \( \$ \) & \( q_{accept} \) & Pop \\
\( q_{reject} \) & \( \epsilon \) & \( \epsilon \) & \( q_{reject} \) & No operation \\
\hline
\end{tabular}
\caption{Таблиця переходів для МА}
\label{tab:transition_table}
\end{table}

Перехід до \( q_{reject} \) відбувається, коли МА зустрічає 'b', коли 'A' знаходиться на вершині стека, або 'a', коли 'B' знаходиться на вершині стека. Це гарантує, що МА точно відображає обмеження мови, приймаючи лише ті рядки, які відповідають правилам парної кількості послідовних 'b' після будь-якої кількості 'a', і навпаки.

\subsection*{Діаграма}
Для кращого розуміння наведено діаграму станів МА. Діаграма ілюструє стани та переходи МА під час обробки вхідного рядка згідно з визначеною граматикою та правилами розпізнавання.

\begin{figure}[h!]
  \centering
  \includegraphics[width=0.9\textwidth]{pda_diagram.png}
  \caption{Діаграма станів МА. (Згенеровано за допомогою коду)}
  \label{fig:pda_diagram}
\end{figure}

\newpage

\section*{Висновки}

У цій роботі ми успішно описали процес побудови МА для заданої контекстно-вільної граматики. МА було формалізовано у вигляді сімки, що включають необхідні компоненти: множину станів, вхідний та стековий алфавіти, функцію переходу, початковий стан, початковий символ стека та множину акцептуючих станів. За допомогою детального опису ми зіставили правила породження граматики з переходами МА, гарантуючи, що автомат точно розпізнає мову, породжену граматикою.

\end{document}
